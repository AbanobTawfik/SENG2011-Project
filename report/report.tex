\documentclass{article}
\usepackage[utf8]{inputenc}
\usepackage[a4paper, total={6in, 8in}]{geometry}
\usepackage{graphicx}
\usepackage{amsmath}
\usepackage{amssymb}
\usepackage{booktabs} % for "\midrule" macro
\usepackage{lipsum} % for filler text
\usepackage{enumerate}
\usepackage{amsmath}
\usepackage{array}
\usepackage{lplfitch}
\usepackage{hyperref}
\usepackage{caption}
\renewcommand{\labelitemii}{$\star$}
\newcommand*\moveToRight[1]{\hspace*{0em plus 1fill}\makebox{(#1)}}
\newcommand*\fixindent{ \hspace{1pt}\\}
%this command below is not my work was used for quality of life
%link to original post 
%https://tex.stackexchange.com/questions/330588/how-to-produce-given-number-of-quad-in-math
\newcommand{\myquad}[1][1]{\hspace*{#1em}\ignorespaces}
\title{SENG2011\\Project Report}        

\author{Abanob Tawfik z5075490\\
        Kevin Luxa z5074984\\
        Lucas Pok z5122535\\
        Michael Yoo z5165635\\
        Rason Chia z5084566
}
\date{October 2019}

\begin{document}

\maketitle
\section{Executive Summary}
Project Vampire is an electronic records management system for the handling of medical blood inventories.

\fixindent{}The project aims to replace paper-based records systems, with the usual benefits of digital records management - such as reducing costs and improving usability. However, there are some concerns as to whether a digital system can be relied upon to manage a critical resource with human lives at stake. To provide assurance, our system uses formal verification techniques to ensure that it conforms to safety requirements.

\fixindent{}Our system integrates with pre-existing business processes of blood inventory management. For example, there are built-in processes for the expiration of blood, distinction between blood types, requests and acceptance of donated blood, to list a few.

\fixindent{}Our system comprises of a graphical interface that interacts with a verified data store service. The implementation details will be detailed in their respective sections.

\fixindent{}Our final project and report will describe a minimum viable product to demonstrate that safety and assurance requirements can be met with a digital records management system.

\newpage

\section{Requirements}
\subsection{Stakeholders}
\begin{itemize}
    \item \textbf{Blood inventory staff [Vampire employee]}
        \begin{itemize}
            \item Handles the physical delivery of requested blood to hospitals.
            \item Physically disposes of expired blood by sending it to a blood disposal service.
        \end{itemize}
    \item \textbf{Hospital staff}
        \begin{itemize}
            \item Can request blood to treat patients.
            \item Can query the level of different types of blood in Vampire’s inventory.
        \end{itemize}
    \item \textbf{Batmobile admin [Vampire employee]}
        \begin{itemize}
            \item Deposits blood into the system.
        \end{itemize}
    \item \textbf{Pathology staff}
        \begin{itemize}
            \item Tests all blood within the system.
        \end{itemize}
    \item \textbf{Blood donor}
        \begin{itemize}
            \item Donates blood to the Batmobile.
            \item Can see the outcome of their blood donation.
        \end{itemize}
    \item \textbf{Emergency donor}
        \begin{itemize}
            \item Upon hitting a critically low supply of blood in the system, these people will donate blood via the Batmobile.
        \end{itemize}
    \item \textbf{Headquarters}
        \begin{itemize}
            \item Business owners.
            \item Manage all critical and sensitive actions.
        \end{itemize}
    \item \textbf{Blood dump}
        \begin{itemize}
            \item Will eliminate blood in a medically suitable fashion.
        \end{itemize}
    \item \textbf{Blood donation facility}
        \begin{itemize}
            \item Will work in conjunction with the Batmobile to offer a venue in which blood donors  are able to visit to donate blood.
        \end{itemize}
\end{itemize}

\newpage
\subsection{Priority Description}
\begin{itemize}
    \item \textbf{P1 (Priority 1, Must have)}
        \begin{itemize}
            \item All these requirements form the minimum viable product, crucial to launch.
        \end{itemize}
    \item \textbf{P2 (Priority 2, Should have)}
        \begin{itemize}
            \item All these requirements should be done in the time frame, but are not critical to launch, can be delayed for future release.
        \end{itemize}
    \item \textbf{P3 (Priority 3, Could have)}
        \begin{itemize}
            \item All these requirements are features we would like, but in the time frame we may delay them to future releases to focus on priority 1 and priority 2.
        \end{itemize}
    \item \textbf{P4 (Priority 4, Would have)}
        \begin{itemize}
            \item Features we do not expect to have but show the road map of where we see our product down the line.
        \end{itemize}
\end{itemize}

\subsection{Requirements List}
\begin{enumerate}
    \item \textbf{Hospital staff are able to query the inventory [P1]}
\end{enumerate}
\end{document}
